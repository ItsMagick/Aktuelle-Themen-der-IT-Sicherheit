%! Author = charon
%! Date = 11/21/24

\section{Hintergrund}\label{sec:hintergrund}
Das Fuzzing besteht aus mehreren Komponenten und Herangehensweisen.
Diese Methodiken und Strategien sind entscheidend für die Effektivität des Fuzzings und die
Identifizierung von Schwachstellen.
Im Folgenden werden die wesentlichen und Bereiche des Fuzzings erläutert und die Limitationen des Fuzzings aufgezeigt.
\subsection{Fuzzing Herangehensweisen}\label{subsec:testing-methoden}
Es gibt drei grundlegende Herangehensweisen an das Fuzzing: Black Box, White Box und Grey Box Fuzzing.
\begin{itemize}
    \item Black Box Fuzzing
    \item White Box Fuzzing
    \item Grey Box Fuzzing
\end{itemize}
\subsection{Sparten des Fuzzing}\label{subsec:sparten-des-fuzzing}
Mit der Verbreitung von Fuzzing-Techniken in verschiedenen Anwendungsbereichen haben sich spezialisierte Sparten des Fuzzings
entwickelt, die auf die spezifischen Anforderungen und Herausforderungen dieser Bereiche zugeschnitten sind.
Für eine modulare Anwendung von Fuzzing-Techniken wurden bereits erste Bibliotheken und Frameworks entwickelt, die
die Integration von Fuzzing in verschiedene Anwendungsbereiche erleichtern.
Ein etabliertes Beispiel ist das American Fuzzy Lop (AFL)~\cite{afl} Fuzzing Framework mit der Weiterentwicklung AFL++~\cite{AFLplusplus-Woot20}.
AFL++ wurde anschließend in der Programmiersprache Rust neu implementiert und als Bibliothek zur Entwicklung von Fuzzern
unter dem Namen LibAFL~\cite{fioraldi_libafl_2022} veröffentlicht.
\begin{enumerate}
    \item \textit{IoT Fuzzing}: Das Fuzzing im Internet der Dinge (IoT) konzentriert sich auf die Sicherheitsprüfung von IoT-Geräten, die aufgrund ihrer
    spezifischen Merkmale und ressourcenbedingten Einschränkungen häufig anfällig sind.
    Die Sicherheitsherausforderungen im IoT unterscheiden sich von denen traditioneller IT-Systeme, da IoT-Geräte
    oftmals über keine ausreichenden Sicherheitsmechanismen verfügen~\cite{eceiza_fuzzing_2021}.
    \item \textit{Firmware-Fuzzing}: Firmware-Fuzzing zielt auf die Firmware von Geräten ab, die als Low-Level-Software die Hardware-Komponenten steuert.
    Diese Methode ist besonders wichtig, da Firmware oft Schwachstellen enthält, die von Angreifern ausgenutzt werden können.
    Die Effektivität des Fuzzings in diesem Kontext hängt von der Fähigkeit ab, Eingaben zu generieren, die mit den Funktionen
    der Firmware interagieren.
    Zu den Herausforderungen zählen die notwendige Kenntnis der Firmware-Struktur sowie ressourcenbedingte Einschränkungen,
    die den Fuzzing-Prozess limitieren können~\cite{eceiza_fuzzing_2021}.
    \item \textit{Binary-Fuzzing}: Binary-Fuzzing konzentriert sich auf die Analyse von kompilierten Binärdateien anstelle von Quellcode.
    Diese Technik ist insbesondere für proprietäre Software ohne zugänglichen Quellcode von Bedeutung.
    Hierbei werden zufällige oder semi-zufällige Eingaben an die Binärdateien übermittelt, um Abstürze oder unerwartetes Verhalten
    zu detektieren.
    Die Herausforderung besteht in der fehlenden Transparenz über die interne Funktionsweise der Binärdateien, was die Erstellung
    effektiver Testfälle erschwert.
    Dennoch können durch Binary-Fuzzing kritische Schwachstellen aufgedeckt werden, die mit anderen Testmethoden möglicherweise
    unentdeckt bleiben~\cite{eceiza_fuzzing_2021}.
    \item \textit{Netzwerkprotokoll-Fuzzing}: Das Netzwerkprotokoll-Fuzzing dient der Prüfung der Robustheit von Netzwerkprotokollen, indem fehlerhafte oder unerwartete
    Pakete an einen Netzwerkdienst gesendet werden.
    Diese Art des Fuzzings ist entscheidend, um Schwachstellen in Kommunikationsprotokollen zu identifizieren, die von
    Angreifern ausgenutzt werden könnten, um unbefugten Zugriff zu erlangen oder Dienste zu stören.
    Für die effektive Generierung von Testfällen ist ein Verständnis der Protokollspezifikationen unerlässlich, wie die Arbeit
    betont~\cite{eceiza_fuzzing_2021}.
    \item \textit{Betriebssystem-Fuzzing}: Das Betriebssystem-Fuzzing befasst sich mit der Sicherheitsprüfung des Betriebssystems selbst, um Schwachstellen zu identifizieren,
    die von Schadsoftware oder böswilligen Nutzern ausgenutzt werden könnten.
    Diese Technik ist von zentraler Bedeutung, da Sicherheitslücken auf Betriebssystemebene schwerwiegende Konsequenzen wie
    unbefugten Zugriff oder Datenverluste nach sich ziehen können.
    Zu den Herausforderungen zählen die hohe Komplexität der Betriebssystemarchitektur sowie die Notwendigkeit, eine umfassende
    Abdeckung der verschiedenen Systemkomponenten zu gewährleisten~\cite{eceiza_fuzzing_2021}.
\end{enumerate}
\subsection{Limitationen des Fuzzing}\label{subsec:limitationen-des-fuzzign}
Fuzzing hat einige Limitationen, die seine Effektivität einschränken können: \newline
\textit{Lückenhafte Codeabdeckung}:
Komplexe Programme mit tief verschachtelten Bedingungen oder umfangreichen Verzweigungen können dazu führen, dass bestimmte
Codepfade ungetestet bleiben.
Dies resultiert aus der Schwierigkeit, Eingaben zu generieren, die alle möglichen Pfade durchlaufen.
Zudem können Techniken wie Code-Obfuskation die Effektivität von Fuzzern beeinträchtigen, indem sie die Pfadanzahl
künstlich erhöhen und somit die Analyse erschweren~\cite{wachter_fuzzing}.\newline
\textit{Unzureichende Testfälle}:
Die Qualität der durch Fuzzing generierten Testfälle hängt von der eingesetzten Strategie ab.
Black-box-Fuzzing kann ineffizient sein und wichtige Fehler übersehen.
Obwohl Gray-box-Fuzzing durch Instrumentierung versucht, ein Gleichgewicht zu finden, bleibt die Herausforderung
bestehen, für alle Szenarien -- vor allem bei umfangreichen Systemen -- zureichende Testfälle zu generieren.\newline
\textit{Verstehen von komplexen Codepfaden}:
Komplexe Codepfade, insbesondere solche mit tiefen Verschachtelungen oder zahlreichen Bedingungen, stellen für Fuzzer eine
Herausforderung dar.
Die Pfadexplosion, also die exponentielle Zunahme möglicher Pfade, kann dazu führen, dass nicht alle relevanten Pfade
effektiv getestet werden.
Obwohl hybride Ansätze, die Fuzzing mit symbolischer Ausführung kombinieren, entwickelt wurden, um diese Hürde zu beheben,
bleibt die vollständige Abdeckung komplexer Pfade eine Herausforderung~\cite{noller_badger_2018}.\newline
\textit{Nichtdeterministische Programme und zustandsabhängige Systeme}:
Programme, deren Verhalten von internen Zuständen oder externen Faktoren abhängt, erschweren die Anwendung von Fuzzing.
Nichtdeterministisches Verhalten kann dazu führen, dass identische Eingaben unterschiedliche Ergebnisse produzieren, was
die Reproduzierbarkeit von Tests beeinträchtigt.
Zudem können zustandsabhängige Systeme, die auf spezifischen Sequenzen von Eingaben reagieren, von traditionellen
Fuzzing-Ansätzen nur unzureichend getestet werden.
Obwohl fortschrittliche Fuzzer versuchen, Zustandsinformationen zu integrieren, bleibt die effektive Handhabung
nichtdeterministischer und zustandsabhängiger Systeme eine offene Forschungsfrage~\cite{she_fox_2024,pham_aflnet_2020}.

