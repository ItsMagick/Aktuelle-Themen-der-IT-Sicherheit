%! Author = charon
%! Date = 11/21/24

\section{Konklusion}\label{sec:konklusion}
Fuzzing ist ein Werkzeug für die Sicherheitsanalyse von Software.
Mit der Integration von künstlicher Intelligenz und Machine Learning kann das Potenzial dieser Technik erweitert werden.
Neue Methoden wie der Einsatz von LLMs, Reinforcement Learning und Generative Adversarial Networks verbessern die Qualität
und Vielfalt der generierten Testfälle.
Dennoch existieren weiterhin Herausforderungen, wie die hohe Rechenintensität von KI-Modellen und die Komplexität von
tief verschachtelten Codepfaden.
Diese Herausforderungen verdeutlichen den Bedarf an Forschungen, unter anderem im Bereich automatisierter und
skalierbarer Fuzzing-Ansätze.
Die Kombination traditioneller Techniken mit innovativen KI-basierten Ansätzen hat das Potenzial, die Schwachstellenerkennung
in Software fundamental zu verändern und die Sicherheit von Anwendungen nachhaltig zu verbessern.