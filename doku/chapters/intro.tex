%! Author = charon
%! Date = 11/20/24

\section{Einführung}\label{sec:introduction}
Fuzzing ist eine Technik zur Identifikation von Sicherheitslücken und Programmfehlern in Software.
Durch die automatisierte Generierung und Überprüfung von Eingabedaten werden Schwachstellen identifiziert.
Im Kern basiert Fuzzing auf der systematischen oder zufälligen Erzeugung von Testdaten, die an das System unter Test (SUT) übermittelt
werden, um dessen Verhalten unter außergewöhnlichen Bedingungen zu analysieren.
Die Ergebnisse werden genutzt, um Abstürze, Speicherfehler oder andere unerwünschte Zustände zu detektieren.\newline
Die Integration von Künstlicher Intelligenz (KI) und Machine Learning (ML) in das Fuzzing kann zur Verbesserung der
Effizienz und Effektivität durch Automatisierung von Entscheidungsprozessen und Optimierung der Testdatengenerierung führen.
LLMs wie GPT oder BERT~\cite{devlin_bert_2019} haben sich als leistungsfähig bei der Analyse und Generierung von natürlicher Sprache und
strukturierten Daten erwiesen.\newline
Im Kontext von Fuzzing können LLMs genutzt werden, um semantisch sinnvolle Testeingaben für Anwendungen zu generieren,
die stark strukturierte oder grammatikabhängige Eingaben erwarten~\citet{deng_large_2023}.
Durch ihre Fähigkeit syntaktische und semantische Abhängigkeiten in komplexen Eingabeformaten zu verstehen können LLMs helfen
die Effizienz des Fuzzing-Prozesses zu steigern.\newline
Auch Generative Adversarial Networks (GAN) können im Fuzzing eingesetzt werden, um realistische Testeingaben zu generieren.
GANs bestehen aus einem Generator- und einem Diskriminator-Modell.
Im Fuzzing-Umfeld können sie eingesetzt werden, um Eingabedaten zu erzeugen, die spezifische Eigenschaften oder Schwachstellen
adressieren.
Beispielsweise könnte der Generator synthetische Testfälle erzeugen, die mögliche Schwachstellen in der Eingabeverarbeitung
eines Programms explorieren, während der Diskriminator -- in diesem Kontext das SUT mit Monitoring-Mechanismen -- die Qualität
der erzeugten Eingaben bewertet~\citet{devlin_bert_2019}.
Dies erlaubt die Generierung diverser und relevanter Testfälle, insbesondere für Programme mit komplexen Anforderungen.\newline
Im Kontext des Fuzzings kann Reinforcement Learning (RL) genutzt werden, um die Sequenz von Eingaben oder Teststrategien zu optimieren.
Hierbei wird ein Agent trainiert, der durch Interaktionen mit dem Zielprogramm lernt, Eingaben zu generieren, die mit
höherer Wahrscheinlichkeit Schwachstellen auslösen.
Belohnungsfunktionen können beispielsweise auf Metriken wie Codeabdeckung, Anomalien im Speicherzugriff oder spezifischen
Abstürzen basieren~\cite{paduraru_riverfuzzrl_2021}.\newline
Die Kombination von Fuzzing-Methoden mit KI-gestützten Ansätzen bietet Potenziale.
Beispielsweise können durch die Nutzung von LLMs, GANs und RL neue Eingabemuster erschlossen werden, die mit traditionellen Techniken
nur schwer zu herauszufinden.
Existierend Herausforderungen, wie die Skalierbarkeit der Ansätze bei hochkomplexen Programmen, die hohe Rechenintensität von
ML-Modellen sowie die Erklärbarkeit und Vertrauenswürdigkeit der generierten Ergebnisse bleiben dennoch bestehen und
stellen ein weiteres Forschungsproblem in der Zukunft dar~\cite{ramadan_role_2024}.