%! Author = charon
%! Date = 11/20/24

\section{Einführung}\label{sec:introduction}
Fuzzing ist eine automatisierte Testmethode, die darauf abzielt, Schwachstellen in Software durch die Generierung vielfältiger
und unerwarteter Eingaben aufzudecken.
Diese Technik ist besonders effektiv bei der Identifizierung von Abstürzen oder Verletzungen von Assertions in Programmen.
Im Kontext cyber-physischer Systeme (Cyber-Physical Systems, CPSs) kann Fuzzing auf Netzwerkebene angewendet werden,
indem Aktoren mit Befehlen adressiert werden, die nicht von den speicherprogrammierbaren Steuerungen
(Programmable Logic Controllers, PLCs) stammen.