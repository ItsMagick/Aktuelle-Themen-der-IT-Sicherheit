%! Author = chaorn
%! Date = 17.01.25

\section{Verwandte Arbeiten}\label{sec:related-work}
Zu dem Thema Anwendungsgebiete für ML-Techniken in Fuzzing gibt es bereits einige Arbeiten, die sich mit der Thematik auseinandersetzen.
\citet{saavedra_review_2019} untersuchen die Integration von ML in Fuzzing-Prozesse und heben dabei die Anwendungen in der
Eingabegenerierung und der Post-Fuzzing-Analyse hervor.
Gleichzeitig identifizieren sie Herausforderungen wie Datenbeschränkungen und die rechnerische Komplexität.\newline
\citet{wang_systematic_2020} liefern eine systematische Übersicht über den Einsatz von ML-Techniken im Fuzzing.
Sie diskutieren verschiedene Phasen, in denen ML angewendet werden kann, die verwendeten Algorithmustypen sowie die
Leistungsmetriken zur Bewertung dieser Methoden.\newline
\citet{zhong_neural_2022} stellen mit AutoFuzz einen grammatikgesteuerten Ansatz für das Fuzzing von autonomen Fahrzeugen
vor, der auf ML basiert.
AutoFuzz kombiniert evolutionäre Algorithmen mit neuronalen Netzen, um komplexe Fahrszenarien in Simulationen zu generieren,
die zu vielfältigen Verkehrsverstößen führen können.
Die Ergebnisse zeigen, dass AutoFuzz im Vergleich zu Baseline-Methoden \num{10}\text{--}\num{39}\,\unit{\percent}
mehr einzigartige Verkehrsverstöße entdeckt und somit die Robustheit von AV-Software verbessern kann.\newline
\citet{she_mtfuzz_2020} führen mit MTFuzz ein neuartiges Fuzzing-Framework ein, das auf Multi-Task Neuronalen Netzwerken basiert.
Durch die Nutzung eines kompakten Embeddings und Gradient-basierter Mutationen verbessert MTFuzz die Abdeckung von
Programmverzweigungen.
Es deckt bis zu dreimal mehr Kanten ab als aktuelle state-of-the-art Fuzzer und identifiziert dabei elf zuvor unbekannte
Softwarefehler.\newline
\citet{chen_learning-guided_2019} entwickeln eine ML-gestützte Fuzzing-Technik für Cyber-Physical Systems (CPS), die intelligente
Angriffe auf Netzwerke von Aktoren simuliert, um unsichere physische Zustände zu erzeugen.
Mithilfe eines prädiktiven Modells und heuristischer Suche generiert diese Methode Testfälle, die spezifische
Sicherheitsmechanismen von CPS prüfen.
In zwei realen CPS-Testbeds identifizierte der Ansatz 27 unsichere Zustände, darunter sechs, die in etablierten Benchmarks
nicht enthalten waren.