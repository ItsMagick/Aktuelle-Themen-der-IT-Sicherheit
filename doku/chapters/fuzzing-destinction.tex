%! Author = charon
%! Date = 11/21/24

\section{Unterscheidung von Fuzzing-Ansätzen}\label{sec:unterscheidung-von-fuzzern}
Im Folgenden werden die wesentlichen Aspekte der Fuzzing-Ansätze, basierend auf Eingabegenerierung, Intelligenz der Fuzzer
und Explorationsstrategien, wie sie in der Arbeit von \citet{eceiza_fuzzing_2021} beschrieben sind, detailliert erläutert.
\subsection{Eingabegenerierung}\label{subsec:input-generation}
Die Eingabegenerierung ist entscheidend für die Effektivität des Fuzzings und kann in zwei Hauptansätze unterteilt werden:
\begin{itemize}
    \item Mutationsbasierte Ansätze: Diese Ansätze generieren neue Eingaben, indem bestehende Testfälle (sogenannte Seeds) 
    verändert werden. 
    Die Qualität des initialen Seeds ist dabei von zentraler Bedeutung, da sie direkt die Abdeckung und Wirksamkeit des 
    Fuzzing-Prozesses beeinflusst. 
    Die Mutation umfasst Änderungen an der ursprünglichen Eingabe, ohne dass Kenntnisse über die Eingabespezifikationen 
    erforderlich sind, was sie für komplexe Eingabeformate geeignet macht. 
    Allerdings kann die kontinuierliche Zunahme der Seed-Größe die Geschwindigkeit des Fuzzers beeinträchtigen. 
    Beispiele für mutationsbasierte Fuzzer sind AFL~\cite{afl} und Angora~\cite{chen_angora_2018}.
    \item Generierungsbasierte Methoden: Dieser Ansatz setzt Kenntnisse über die Eingabespezifikationen und Protokolle 
    des SUT voraus.
    Er ist besonders wichtig, wenn gültige Eingaben für erfolgreiche Tests erforderlich sind. 
    Generierungsbasierte Fuzzer erstellen Eingaben basierend auf vordefinierten Formaten, was zu effizienteren Tests führt, 
    insbesondere wenn die Struktur der Eingaben eingeschränkt ist.
\end{itemize}
\subsection{Intelligenz des Fuzzers}\label{subsec:intelligenz-des-fuzzers}
Die Intelligenz eines Fuzzers beschreibt seine Fähigkeit, die Generierung von Eingaben basierend auf Rückmeldungen aus
vorherigen Testausführungen anzupassen.
Fuzzer können hierbei wie folgt eingeteilt werden:
\begin{itemize}
    \item Intelligente Fuzzer (Smart Fuzzers): Diese passen ihre Strategien zur Eingabegenerierung auf Basis des Feedbacks
    des SUT an.
    Sie lernen aus den Ausgaben und dem Verhalten des SUT, wodurch sie gezielt Testfälle generieren können.
    Intelligente Fuzzer sind in der Regel effizienter bei der Entdeckung von Schwachstellen, da sie weniger Testfälle benötigen.
    Beispiele hierfür sind Learn\&Fuzz~\cite{godefroid2017learnfuzzmachinelearninginput} und IoTFuzzer~\cite{noauthor_iotfuzzer_nodate}.
    \item Einfach gestrickte Fuzzer (Dumb Fuzzers): Im Gegensatz dazu nutzen dumb Fuzzer keine Rückmeldungen für die
    Generierung neuer Eingaben.
    Obwohl sie Tests schneller ausführen und generieren können, sind sie weniger effektiv bei der Entdeckung von Schwachstellen.
    Ein Beispiel für einen solchen Fuzzer ist zzuf~\cite{zzuf}.
\end{itemize}
\subsection{Explorationsstrategien}\label{subsec:explorationsstrategien}
Explorationsstrategien umfassen Methoden, mit denen Fuzzer während des Tests die Codeabdeckung maximieren können. 
Sie lassen sich wie folgt kategorisieren:
\begin{itemize}
    \item Abdeckungsbasierte Fuzzer (Coverage-based Fuzzers): Diese verwenden Analysetechniken, um eine maximale Abdeckung
    verschiedener Codepfade zu erreichen.
    Fuzzer mit hoher Abdeckung sind effektiver bei der Fehlerfindung, da sie mit minimalem Aufwand möglichst viele
    Ausführungspfade erkunden.
    Ein Beispiel für einen abdeckungsbasierten Fuzzer ist SAGE~\cite{godefroid_sage_2012}.
    \item Direktive Fuzzer (Directed Fuzzers): Diese konzentrieren sich auf spezifische Teile des Codes, was zu
    schnelleren Ergebnissen führt, da sie relevante Bereiche, wie kürzlich geänderten Code oder kritische Anwendungsbereiche,
    gezielt adressieren.
    Direktive Fuzzer nutzen häufig symbolische Ausführung, um ihre Eingabegenerierung zu leiten~\cite{liang_sequence_2019}.
\end{itemize}
