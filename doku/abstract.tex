%! Author = charon
%! Date = 11/20/24
\begin{abstract}
    Fuzzing ist eine Technik zur Identifizierung von Sicherheitslücken und Programmfehlern in Software.
    Dieser Überblick untersucht die Grundlagen, Herangehensweisen und Anwendungsgebiete von Fuzzing, wie IoT-, Firmware-,
    Netzwerkprotokoll- und Betriebssystem-Fuzzing.
    Dabei werden die Herausforderungen dieser Technik, wie unzureichende Codeabdeckung und Probleme mit nichtdeterministischen Programmen,
    erläutert.
    Zudem wird der zunehmende Einsatz von Machine-Learning-Methoden im Fuzzing untersucht.
    Ansätze wie generative KI, Reinforcement Learning und genetische Algorithmen optimieren die Eingabegenerierung und
    Teststrategien, während Techniken wie dynamic symbolic execution und guided coverage die Effizienz und Effektivität
    verbessern.
    Trotz der Fortschritte bleibt die Integration von KI-basierten Ansätzen ein aktives Forschungsfeld,
    insbesondere im Hinblick auf die Skalierbarkeit und Erklärbarkeit der generierten Ergebnisse.
\end{abstract}