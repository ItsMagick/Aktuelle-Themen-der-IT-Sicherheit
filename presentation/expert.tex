%! Author = chaorn
%! Date = 11.12.24

\section{Machine Learning und Fuzzing}\label{sec:machine-learning-und-fuzzing}
\begin{frame}{Machine Learning und Fuzzing}
    Wieso Machine Learning und Fuzzing?
    \begin{itemize}
        \item Machine Learning kann interessante Eingaben vorhersagen
        \item Komplexe Strukturen von Eingaben können erkannt werden
        \item Verbessertes Verständnis von Code(-fehl-)verhalten
    \end{itemize}
\end{frame}
\begin{frame}{Begin von Machine Learning und Fuzzing}
    \begin{figure}[H]
        \centering
        \includegraphics[width=\textwidth]{res/pulsar}
        \caption[Pioniere des Fuzzing mit Machine Learning]{
            Veröffentlichung des ersten Fuzzers mit statistischen Modellen~\cite{thuraisingham_pulsar_2015}
        }
        \label{fig:ml_fuzzing}
    \end{figure}
\end{frame}
\begin{frame}{Pulsar}
    \begin{itemize}
        \item Erster Fuzzer mit \enquote{Machine Learning}
        \item Verwendung von Markov-Modellen
        \item Erkennt und simuliert Zustände
        \item Erkennt und simuliert Nachrichten
        \item Kann sowohl ein Protokoll simulieren, als auch das tatsächliche Protokoll fuzzen
    \end{itemize}
\end{frame}
\section{Demo}\label{sec:demo}
\begin{itemize}
    \item Role of Convolutional Neural Networks in fuzzing:
    \begin{itemize}
        \item Analyzing code structure and coverage patterns.
        \item Input prioritization using learned models.
    \end{itemize}
        \item Feature extraction from execution traces and call graphs.
        \item Example application: Predicting program branches that require exploration.
\end{itemize}

\begin{itemize}
    \item Challenges in using ML for fuzzing:
    \begin{itemize}
        \item High computational requirements.
        \item Training data scarcity for certain domains.
    \end{itemize}
    \item Future possibilities:
    \begin{itemize}
        \item Reinforcement learning for fuzzing optimization.
        \item Transfer learning to adapt fuzzers to new domains.
    \end{itemize}
\end{itemize}
