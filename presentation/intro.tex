%! Author = charon
%! Date = 12/9/24

\section{Einführung}\label{sec:einfuhrung}
\begin{frame}{Was ist Fuzzing?}
    \blockquote{\enquote{Fuzzing is a vulnerability discovery solution that resonates with random-mutation, feedback-driven,
        coverage-guided, constraint-guided, seed-scheduling, and target-oriented strategies.
        Each technique is wrapped beneath the black-, white-, and grey-box fuzzers to uncover diverse vulnerabilities.}Sanoop et al.~\cite{fuzzing_methods}}
\end{frame}
\begin{frame}{Was ist Fuzzing?}
    \begin{itemize}
        \item Fuzzing ist eine Methode zur Entdeckung von Schwachstellen in Software.
        \item Es gibt verschiedene Arten von Fuzzing-Methoden.
        \item Methoden: Black-, White- und Grey-Box Fuzzing.
    \end{itemize}
\end{frame}
\begin{frame}{Fuzzing Methoden}
    \begin{itemize}
        \item Black-Box Fuzzing: Keine Kenntnis über den Quellcode.
        \item White-Box Fuzzing: Kenntnis über den Quellcode.
        \item Grey-Box Fuzzing: Teilweise Kenntnis über den Quellcode.
    \end{itemize}
\end{frame}
%TODO : Missverständnis von Testing zu Fuzzing klären.
\begin{frame}{Fuzzing Anwendungsgebiete}
    \begin{itemize}
        \item Binary Fuzzing
        \item Betriebssystem Fuzzing
        \item Netzwerkprotokoll Fuzzing
        \item Firmware Fuzzing
        \item IoT/Embedded Fuzzing
        \item \ldots
    \end{itemize}
\end{frame}
