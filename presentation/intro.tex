%! Author = charon
%! Date = 12/9/24

\section{Einführung}\label{sec:einfuhrung}
\begin{frame}{Was ist Fuzzing?}
    \blockquote{\enquote{Fuzzing is a vulnerability discovery solution that resonates with random-mutation, feedback-driven,
        coverage-guided, constraint-guided, seed-scheduling, and target-oriented strategies.
        Each technique is wrapped beneath the black-, white-, and grey-box fuzzers to uncover diverse vulnerabilities.}Sanoop et al.~\cite{fuzzing_methods}}
\end{frame}
\begin{frame}{Was ist Fuzzing?}
    \begin{itemize}
        \item Fuzzing ist eine Methode zur Entdeckung von Schwachstellen in Software
        \item Es gibt verschiedene Arten von Fuzzing-Methoden
        \item Methoden: Black-, White- und Grey-Box Fuzzing
    \end{itemize}
\end{frame}
\begin{frame}{Fuzzing Methoden}
    \begin{itemize}
        \item Black-Box Fuzzing: Keine Kenntnis über den Quellcode
        \item White-Box Fuzzing: Kenntnis über den Quellcode
        \item Grey-Box Fuzzing: Teilweise Kenntnis über den Quellcode
    \end{itemize}
\end{frame}
%TODO : Missverständnis von Testing zu Fuzzing klären.
\begin{frame}{Wieso Fuzzing?}
    Fuzzing ist eine:
    \begin{itemize}
        \item Effektive Methode zur Entdeckung von Schwachstellen
        \item Automatisierte Methode
        \item Schnelle(-re?) Methode
        \item Kostengünstige Methode
    \end{itemize}
\end{frame}
\begin{frame}{Fuzzing Anwendungsgebiete}
    \begin{itemize}
        \item Binary Fuzzing
        \item Betriebssystem Fuzzing
        \item Netzwerkprotokoll Fuzzing
        \item Firmware Fuzzing
        \item IoT/Embedded Fuzzing
        \item \ldots
    \end{itemize}
\end{frame}
\begin{frame}{Wie Funktionert Fuzzing?}
    \begin{tikzpicture}[node distance=2.5cm]
        \node[draw, rectangle, rounded corners, minimum height=1cm, minimum width=2.5cm] (testcase) {\textbf{Initial Testcase}};
        \node[draw, rectangle, rounded corners, minimum height=1cm, minimum width=2.5cm, right of=testcase, xshift=1.8cm] (dryrun) {\textbf{Dry Run}};
        \node[draw, rectangle, rounded corners, minimum height=1cm, minimum width=2.5cm, right of=dryrun, xshift=1.4cm] (mutator) {\textbf{Mutator}};
        \node[draw, rectangle, rounded corners, minimum height=1cm, minimum width=2.5cm, below of=dryrun] (program) {\textbf{Program unter Test}};
        \node[draw, ellipse, minimum height=1cm, minimum width=2.2cm, right of=program, xshift=1.4cm] (feedback) {\textbf{Feedback}};

        \draw[->, thick] (testcase) -- (dryrun);
        \draw[->, thick] (dryrun) -- (mutator);
        \draw[->, thick] (mutator) -- (program);
        \draw[->, thick] (program) -- (feedback);
        \draw[->, thick] (feedback) -- (mutator);
    \end{tikzpicture}
\end{frame}
\begin{frame}{Wie Funktionert Fuzzing?}
    \begin{tikzpicture}[node distance=2.5cm]
        \node[draw, rectangle, rounded corners, minimum height=1cm, minimum width=2.5cm, fill=blue!10] (testcase) {\textbf{Initial Testcase}};
        \node[draw, rectangle, rounded corners, minimum height=1cm, minimum width=2.5cm, right of=testcase, xshift=1.8cm] (dryrun) {\textbf{Dry Run}};
        \node[draw, rectangle, rounded corners, minimum height=1cm, minimum width=2.5cm, right of=dryrun, xshift=1.4cm, fill=blue!10] (mutator) {\textbf{Mutator}};
        \node[draw, rectangle, rounded corners, minimum height=1cm, minimum width=2.5cm, below of=dryrun] (program) {\textbf{Program unter Test}};
        \node[draw, ellipse, minimum height=1cm, minimum width=2.2cm, right of=program, xshift=1.4cm, fill=blue!10] (feedback) {\textbf{Feedback}};

        \draw[->, thick] (testcase) -- (dryrun);
        \draw[->, thick] (dryrun) -- (mutator);
        \draw[->, thick] (mutator) -- (program);
        \draw[->, thick] (program) -- (feedback);
        \draw[->, thick] (feedback) -- (mutator);
    \end{tikzpicture}
\end{frame}
\begin{frame}{Herausforderung von Fuzzing}
    \begin{itemize}
        \item Lückenhafte Codeabdeckung
        \item Unzureichende Testfälle
        \item Verstehen von komplexen Codepfaden
        \item Nichtdeterministische Programme und Zustandsabhängige Systeme
    \end{itemize}
\end{frame}